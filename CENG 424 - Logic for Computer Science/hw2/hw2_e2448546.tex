\documentclass[12pt]{article}
\usepackage[utf8]{inputenc}
\usepackage{float}
\usepackage{amsmath}
\usepackage[tableaux]{prooftrees}
\renewcommand*\linenumberstyle[1]{(#1)}

\usepackage[hmargin=3cm,vmargin=6.0cm]{geometry}
\topmargin=-2cm
\addtolength{\textheight}{6.5cm}
\addtolength{\textwidth}{2.0cm}
\setlength{\oddsidemargin}{0.0cm}
\setlength{\evensidemargin}{0.0cm}
\usepackage{indentfirst}
\usepackage{amsfonts}
\usepackage{tikz}
\usepackage{qtree}

\begin{document}

\section*{Student Information}

Name: Kaan Karaçanta \\

ID: 2448546 \\


% 1. Suppose that you are given the following premises:
% • p ∧ q ⇒ r
% • q ∨ ¬q
% • p
% Show that the sentence r is logically entailed by the supplied premises.

\section*{Answer 1}
% Use truth table method to show that the sentence r is logically entailed by the supplied premises.

\begin{table}[H]
    \centering
    \begin{tabular}{|l|l|l|l|l|}
        \hline
        $p$ & $q$ & $r$ & $ (p \wedge q) \rightarrow r  $ & $ q \vee \neg q $ \\ \hline
        T   & T   & T   & T                               & T                 \\ \hline
        T   & T   & F   & F                               & T                 \\ \hline
        T   & F   & T   & T                               & T                 \\ \hline
        T   & F   & F   & T                               & T                 \\ \hline
        F   & T   & T   & T                               & T                 \\ \hline
        F   & T   & F   & T                               & T                 \\ \hline
        F   & F   & T   & T                               & T                 \\ \hline
        F   & F   & F   & T                               & T                 \\ \hline
    \end{tabular}

    \caption{Truth table for the given premises and the conclusion}
    \label{t1}
\end{table}

\begin{table}[H]
    \centering
    \begin{tabular}{|l|l|l|l|l|}
        \hline
        $p$ & $q$ & $r$ & $ (p \wedge q) \rightarrow r  $ & $ q \vee \neg q $ \\ \hline
        T   & T   & T   & T                               & T                 \\ \hline
        X   & X   & X   & X                               & X                 \\ \hline
        T   & F   & T   & T                               & T                 \\ \hline
        T   & F   & F   & T                               & T                 \\ \hline
        X   & X   & X   & X                               & X                 \\ \hline
        X   & X   & X   & X                               & X                 \\ \hline
        X   & X   & X   & X                               & X                 \\ \hline
        X   & X   & X   & X                               & X                 \\ \hline
    \end{tabular} 

    \caption{Table with the eliminated rows that do not satisfy premises}
    \label{t2}
\end{table}

\begin{table}[H]
    \centering
    \begin{tabular}{|l|l|l|l|l|}
        \hline
        $p$ & $q$ & $r$ & $ (p \wedge q) \rightarrow r  $ & $ q \vee \neg q $ \\ \hline
        T   & T   & T   & T                               & T                 \\ \hline
        X   & X   & X   & X                               & X                 \\ \hline
        T   & F   & T   & T                               & T                 \\ \hline
        X   & X   & X   & X                               & X                 \\ \hline
        F   & T   & T   & T                               & T                 \\ \hline
        X   & X   & X   & X                               & X                 \\ \hline
        F   & F   & T   & T                               & T                 \\ \hline
        X   & X   & X   & X                               & X                 \\ \hline
    \end{tabular}

    \caption{Table with the eliminated rows that do not satisfy the conclusion}
    \label{t3}
\end{table}

Since the set of the remaining rows in the Table 2 is not a subset of the remaining rows in the Table 3, $r$ is not logically entailed by the supplied premises.

% Give a proof of ((p ⇒ ¬r) ⇒ ¬p) from the premises below using Modus Ponens and the standard
% axiom schemata.
% p ⇒ q, q ⇒ r
\section*{Answer 2}

\begin{align*}
    & 1. \hspace{1em} p \Rightarrow q                                                                                    & \text{Premise} \\
    & 2. \hspace{1em} q \Rightarrow r                                                                                    & \text{Premise} \\
    & 3. \hspace{1em} (q \Rightarrow r) \Rightarrow (p \Rightarrow (q \Rightarrow r))                                    & \text{II} \\
    & 4. \hspace{1em} p \Rightarrow (q \Rightarrow r)                                                                    & \text{Modus Ponens: 3, 2} \\
    & 5. \hspace{1em} (p \Rightarrow (q \Rightarrow r)) \Rightarrow ((p \Rightarrow q) \Rightarrow (p \Rightarrow r))    & \text{ID} \\
    & 6. \hspace{1em} (p \Rightarrow q) \Rightarrow (p \Rightarrow r)                                                    & \text{Modus Ponens: 5, 4} \\
    & 7. \hspace{1em} p \Rightarrow r                                                                                    & \text{Modus Ponens: 6, 1} \\
    & 8. \hspace{1em} (p \Rightarrow r) \Rightarrow ((p \Rightarrow \neg r) \Rightarrow \neg p)                          & \text{CR} \\
    & 9. \hspace{1em} (p \Rightarrow \neg r) \Rightarrow \neg p                                                          & \text{Modus Ponens: 8, 7} \\
\end{align*}


% Give a proof of the sentence p from the single premise ¬¬p using only Modus Ponens and the
% standard axiom schemata.
\section*{Answer 3}

\begin{align*}
    & 1. \hspace{1em} \neg\neg p                                                                                                                                                                            & \text{Premise} \\
    & 2. \hspace{1em} (\neg p \Rightarrow \neg p) \Rightarrow ((\neg p \Rightarrow \neg \neg p) \Rightarrow p)                                                                                              & \text{CR} \\
    & 3. \hspace{1em} \neg p \Rightarrow ((\neg p \Rightarrow \neg p) \Rightarrow \neg p)                                                                                                                   & \text{II} \\
    & 4. \hspace{1em} \neg p \Rightarrow (\neg p \Rightarrow \neg p)                                                                                                                                        & \text{II} \\
    & 5. \hspace{1em} (\neg p \Rightarrow ((\neg p \Rightarrow \neg p) \Rightarrow \neg p)) \Rightarrow ((\neg p \Rightarrow (\neg p \Rightarrow \neg p)) \Rightarrow (\neg p \Rightarrow \neg p))          & \text{ID} \\
    & 6. \hspace{1em} (\neg p \Rightarrow (\neg p \Rightarrow \neg p)) \Rightarrow (\neg p \Rightarrow \neg p)                                                                                              & \text{Modus Ponens: 5, 3} \\
    & 7. \hspace{1em} \neg p \Rightarrow \neg p                                                                                                                                                           & \text{Modus Ponens: 6, 4} \\
    & 8. \hspace{1em} (\neg p \Rightarrow \neg \neg p) \Rightarrow p                                                                                                                                      & \text{Modus Ponens: 7, 2} \\  
    & 9. \hspace{1em} \neg \neg p \Rightarrow (\neg p \Rightarrow \neg \neg p)                                                                                                                              & \text{II} \\
    & 10. \hspace{1em} \neg p \Rightarrow \neg \neg p                                                                                                                                                       & \text{Modus Ponens: 9, 1} \\
    & 11. \hspace{1em} p                                                                                                                                                                                    & \text{Modus Ponens: 10, 8} \\                                                                                      
\end{align*}


% Use propositional resolution to prove the following sentence is valid:
% (p ∨ q ⇒ r) ⇒ (p ⇒ (q ⇒ r))
\section*{Answer 4}

% \begin{align*}
%     & \hspace{2em} (p \vee q \Rightarrow r) \Rightarrow (p \Rightarrow (q \Rightarrow r)) \\
%     & \text{I} \hspace{1em} \neg (\neg (p \vee q) \vee r) \vee (\neg p \vee (\neg q \vee r)) \\
%     & \hspace{2em} \neg (\neg (p \vee q) \vee r) \vee (\neg p \vee (\neg q \vee r)) \\
%     & \text{N} \hspace{1em} (\neg \neg (p \vee q) \wedge \neg r) \vee (\neg p \vee (\neg q \vee r)) \\
%     & \hspace{2em} ((p \vee q) \wedge \neg r) \vee (\neg p \vee (\neg q \vee r)) \\
%     & \text{D} \hspace{1em} ((p \vee q) \wedge \neg r) \vee (\neg p \vee \neg q \vee r) \\
%     & \hspace{2em} ((p \vee q) \vee (\neg p \vee \neg q \vee r)) \wedge (\neg r \vee (\neg p \vee \neg q \vee r)) \\
%     & \hspace{2em} (p \vee q \vee \neg p \vee \neg q \vee r) \wedge (\neg r \vee \neg p \vee \neg q \vee r) \\
%     & \text{O} \hspace{1em} \{p, q, \neg p, \neg q, r\} \\
%     & \hspace{2em} \{\neg r, \neg p, \neg q, r\} \\
% \end{align*}

% Since there are literals that are complementary to each other in the same clause, the sentence is valid.

\begin{align*}
    & \hspace{2em} \neg ((p \vee q \Rightarrow r) \Rightarrow (p \Rightarrow (q \Rightarrow r))) \\
    & \text{I} \hspace{1em} \neg (\neg (\neg (p \vee q) \vee r) \vee (\neg p \vee (\neg q \vee r))) \\
    & \text{N} \hspace{1em} \neg (\neg \neg (p \vee q) \wedge \neg r) \wedge \neg (\neg p \vee (\neg q \vee r)) \\
    & \hspace{2em} ((\neg p \wedge \neg q) \vee r) \wedge (p \wedge (q \wedge \neg r)) \\
    & \text{D} \hspace{1em} ((\neg p \vee r) \wedge (\neg q \vee r)) \wedge (p \wedge (q \wedge \neg r)) \\
    & \hspace{2em} (\neg p \vee r) \wedge (\neg q \vee r) \wedge p \wedge q \wedge \neg r \\
    & \text{O} \hspace{1em} \{\neg p, r\} \\
    & \hspace{2em} \{\neg q, r\} \\
    & \hspace{2em} \{p\} \\
    & \hspace{2em} \{q\} \\
    & \hspace{2em} \{r\} \\
    & \hspace{2em} \{\neg r\} \\
    & \hspace{2em} \{\}
\end{align*}

Here, we showed the negative of the given sentence is unsatisfiable. Thus, the sentence is valid.

\newpage

% Use DPLL algorithm to show that the set of clauses: 
% ¬p ∨ q ∨ s, p ∨ s ∨ t, p ∨ s ∨ ¬t, p ∨ s ∨ t, p ∨ ¬s ∨ ¬t, p ∨ ¬s ∨ t, p ∨ q ∨ ¬s, p ∨ ¬q ∨ s
% is satisfiable or not
\section*{Answer 5}

The set of clauses is:

\begin{align*}
    &(\neg p \vee q \vee s) \\
    &(p \vee s \vee t) \\
    &(p \vee s \vee \neg t) \\
    &(p \vee \neg s \vee \neg t) \\
    &(p \vee \neg s \vee t) \\
    &(p \vee q \vee \neg s) \\
    &(p \vee \neg q \vee s)
\end{align*}

In this set of clauses there are neither tautological clauses nor clauses with pure literals. So, we can not simplify the set of clauses. We can not find a unit clause either. So, we have to choose a literal to assign. Let's choose $p$.

After assigning T to $p$, we have:
\begin{align*}
    &(\neg T \vee q \vee s) \\
    &(T \vee s \vee t) \\
    &(T \vee s \vee \neg t) \\
    &(T \vee \neg s \vee \neg t) \\
    &(T \vee \neg s \vee t) \\
    &(T \vee q \vee \neg s) \\
    &(T \vee \neg q \vee s) \\
\end{align*}

Now, after eliminating tautological clauses, and removing $\neg T$ from the first clause as it does not have any effect, we have:
$$ q \vee s $$

After assigning T to $q$ or $s$, it can be seen that this set of clauses are satisfiable.

\end{document}
